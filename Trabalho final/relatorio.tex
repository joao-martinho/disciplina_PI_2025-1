\documentclass{article}
\author{Felipe Karmann, João Souza}
\date{30 de maio de 2025}
\title{Trabalho final: relatório}

\begin{document}

\maketitle

\section{Descrição do problema}

A partir dos artigos escolhidos para servir de base para o nosso trabalho, o problema a ser resolvido é a criação de um programa que detecte padrões e características do terreno em fotos tiradas por satélites, e, com as informações colhidas no estudo de Lacerda, Fonseca e Faria (2017), calcule as probabiliades de alagamento na região retratada.

\section{Montagem da base}

A base de imagens foi criada com fotos de satélite da região de Blumenau, com foco naquelas com histórico significativo de alagamento.

\section{Preparação}

\section{Modelo da rede}

\section{Treinamento}

\section{Classificação e testes}

\section{Código}

\section{Resultados}
